%%% Local Variables:
%%% mode: latex
%%% TeX-master: "<none>"
%%% End:

\title{Audio compression}

\author{Vicente González Ruiz}

\maketitle

\section{Data flow~\cite{sayood2017introduction}}

    \begin{verbatim}
PCM   +---------+        +---------+ PCM
----->| Encoder |------->| Decoder |----->
audio +---------+ stream +---------+ audio'

              audio != audio'
                (usually)
\end{verbatim}

\section{Typical encoder steps}
\begin{enumerate}
\def\labelenumi{\arabic{enumi}.}
\item
  \textbf{Overlaped subband analysis} (usually with
  \href{http://en.wikipedia.org/wiki/Modified_discrete_cosine_transform\%7D\%20(Modified\%20Discrete\%20Cosine\%20Transform)}{MDCT}.
  Goes from the temporal to a frequency domain.
\item
  \textbf{Quantization}. Basically, removes pure signals of low
  amplitude but taking also into account the SAM (pSycho Acoustic Model)
  of the HAS (Human Auditory System). Noise use to be of low power!
\item
  \href{https://vicente-gonzalez-ruiz.github.io/text_compression/}{\textbf{Entropy
  coding}}.
\end{enumerate}

\section{Lossy coding}
\begin{itemize}
\tightlist
\item
  The limitations of human perception are incorporated into the
  compression process through the use of psychoacoustic models. Some of
  these limitations are physiological, based on the machinery of
  hearing. Others are psychological, based on how our brain processes
  auditory stimuli.
\end{itemize}

    \#\#~Overlaped processing

\begin{verbatim}
0              N-1            2N-1            3N-1
+---------------+---------------+---------------+ s[n]
<--------Transform Step--------->
                <---------Transform Step-------->
\end{verbatim}

\begin{itemize}
\item
  Each transform step inputs \(2N\) samples and outputs \(N\) MDCT
  coeficients.
\item
  \(N\) can vary depending on the characteristics of the sound. For
  \emph{complex} sounds without clear armonics (such as a plosive
  sound), shortened windows improve the performance. For \emph{simple}
  sounds (such as a music instrument), large windows are better.
\end{itemize}

\bibliography{audio-coding}
